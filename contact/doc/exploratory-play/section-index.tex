
\section{Introduction}

The topic of this deliverable is ``exploratory play behavior: learning
by doing.''  In the technical annex, it was envisaged that
the artifact (the robot) would at this point be capable of trying a fixed
set of actions on a fixed set of objects, learning about the objects
and actions using the algorithms of D1.5/D1.6 (``vision algorithms for characterizing objects/actions'').
%
In fact, IST already reported work of this nature in D1.5/D1.6, 
and here provide an update of this work (Section~\ref{sect:manipulation}).
%
The robot behavior is organized on the basis of {\em affordance
discovery and exploitation}.  Affordances in this context are the
particular actions which the robot can apply to particular classes
of objects.  The action set is kept simple: tapping and touching.
The object set is also simple: brightly colored foam balls and
bars.  We concentrate on the problem of {\em learning} to exploit 
affordances through autonomous exploration.

In D1.5/D1.6, we merged work on actions and objects into a single
report, since we believe that the mechanisms for perceiving both
need to be tightly integrated.  The current pair of deliverables,
D1.4 (``advanced babbling'') and D1.7 (``exploratory play'')
create a division between work on speech and work on manipulation.
On a practical level, this is appropriate, since the methods we have
developed and reported in the two areas are, so far, distinct.
But this is somewhat at odds to the goals of the project:

\begin{quote}

``It should be emphasized that the primary goal of the project is not
the engineering of a robotic artifact but a scientific one: of finding
common sensorimotor mechanisms between speech and manipulation.''
{\it (CONTACT Technical Annex)}

\end{quote}

UGDIST has begun work that is explicitly designed at being applicable
to both speech and manipulation, and we report it here as
``exploratory play'' and ``learning by doing'' mechanisms that
are potentially useful in both modalities.  Section~\ref{sect:rhythm} reports
a mechanism for participation in rhythmic events, which we see
as a possible precursor to turn-taking behavior.  
Section~\ref{sect:motor-babble}
reports a general form of ``motor babble.''



