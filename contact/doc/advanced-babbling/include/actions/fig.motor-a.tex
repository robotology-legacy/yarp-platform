% ---------------------------------------------------------------------------- %
\begin{figure}
	\centering
	  \subfigure[\label{fig:actions:F5:motor:a:1}]
	  {\includegraphics[width=0.35\textwidth]{include/actions/images/motor_a1.tps}}
	  \hspace{0.05\textwidth}
	  \subfigure[\label{fig:actions:F5:motor:a:2}]
	  {\includegraphics[width=0.35\textwidth]{include/actions/images/motor_a2.tps}}


	  \subfigure[\label{fig:actions:F5:motor:a:3}]
	  {\includegraphics[width=0.35\textwidth]{include/actions/images/motor_a3.tps}}
	  \hspace{0.05\textwidth}
	  \subfigure[\label{fig:actions:F5:motor:a:4}]
	  {\includegraphics[width=0.35\textwidth]{include/actions/images/motor_a4.tps}}


	\caption[F5 Motor Neuron]{\textbf{F5 Motor Neuron.}
	An example of a grasping neuron is shown. 
	It discharges during precision grip, regardless of the effector involved:
	(a) right hand, (b) left hand.
	When the monkey performs whole hand grasping movements the neuron is silent.
	Moreover, the neuron still express effector-response invariance: (c) right
	hand, (d) left hand.
	Reproduced from~\citet{metta.etal:2006}.}
	\label{fig:actions:F5:motor:a}
\end{figure}
% ---------------------------------------------------------------------------- %
