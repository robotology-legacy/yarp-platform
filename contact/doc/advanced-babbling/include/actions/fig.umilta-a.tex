% ---------------------------------------------------------------------------- %
\begin{figure}
	\centering
	  \subfigure[\label{fig:actions:understanding:umilta:a:1}]
	  {\includegraphics[width=0.3\textwidth]{include/actions/images/umilta_a1.tps}}
	  \hspace{0.025\textwidth}
	  % ---------------------------------------------------------------------- %
	  \subfigure[\label{fig:actions:understanding:umilta:a:2}]
	  {\includegraphics[width=0.3\textwidth]{include/actions/images/umilta_a2.tps}}


	  \subfigure[\label{fig:actions:understanding:umilta:a:3}]
	  {\includegraphics[width=0.3\textwidth]{include/actions/images/umilta_a3.tps}}
	  \hspace{0.025\textwidth}
	  % ---------------------------------------------------------------------- %
	  \subfigure[\label{fig:actions:understanding:umilta:a:4}]
	  {\includegraphics[width=0.3\textwidth]{include/actions/images/umilta_a4.tps}}


	\caption[Mirror neuron: action recognition]{\textbf{Mirror neuron: action recognition.}
	In figure (a) and (b) the experimenter hand starts from a fixed position and moves 
	toward an object.  
	In figure (c) and (d) the object is absent and the experimenter mimics the
	grasping action.
	Figure (a) and (c): full vision condition. Figure (b) and (d): hidden
	condition.
	Reproduced from~\citet{umilta.etal:2001}.}
	\label{fig:actions:understanding:umilta:a}
\end{figure}
% ---------------------------------------------------------------------------- %
