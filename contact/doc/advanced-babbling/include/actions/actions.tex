% ---------------------------------------------------------------------------- %
\chapter{Action representation}
\label{ch:actions}
% ---------------------------------------------------------------------------- %

% ---------------------------------------------------------------------------- %
\section{Monkey rostroventral premotor area}
\label{sec:actions:F5}
% ---------------------------------------------------------------------------- %
Monkey area F5 is located in the ventro-rostral part of inferior premotor area 
6, just caudal to the lower arm of the arcuate sulcus 
(Figure~\ref{fig:actions:mirror:brain-sx}).
Electrical micro-stimulation and single unit recordings show that F5 neurons
discharge during planning/execution of hand and mouth
movements~\citep{rizzolatti.etal:1988}.
F5 has a rough somatotopic organization. 
Hand movements are mostly represented in its dorsal part, whereas mouth
movements are mostly located in the ventral
part~\citep{rizzolatti.etal:1996,rizzolatti.fadiga:1998}.

All F5 neurons share similar motor properties. 
Hand movement F5 neurons have both motor and sensory properties.
Regarding the motor properties, two are the main characteristics. 
Firstly, most neurons discharge during the execution of particular hand-related
movements. Secondly, many of them are specific for particular hand prehension
configurations.
With respect to the sensory properties, the most considerable part of F5 neurons
fire when three dimensional objects are presented.
Moreover, the discharge occurs only when a match between the presented object
and a particular kind of grip occurs.

The properties of F5 neurons controlling hand movements were extensively
studied, while less is known about mouth neurons.
This Section aims to illustrate the properties of F5 neurons, illustrating 
the results of over a decade of neurophysiological investigation. 
For this reason the dissertation mainly focuses on hand neurons; however at the
end of the Chapter the properties of mouth neurons are
discussed~\citep{ferrari.etal:2003}.
% ---------------------------------------------------------------------------- %
\begin{figure}[htbp]
	\centering
		\epsfig{file=include/actions/images/brain_sx.tps, width=0.65\textwidth}

	\caption[Lateral view of macaque monkey brain.]{\textbf{Lateral view of macaque monkey
	brain.}
	The shaded area shows the anatomical localization of the cells recorded by
	Rizzolatti and colleagues~\citep{rizzolatti.etal:1988}. 
	AIP, anterior intraparietal area;
	AIs, inferior arcuate sulcus;
	ASs, superior arcuate sulcus; 
	Cs, central sulcus; 
	IPs, intraparietal sulcus; 
	LIP, lateral intraparietal area; 
	Ls, lateral sulcus; 
	MIP, medial intraparietal area; 
	Ps, principal sulcus; 
	SI, primary somatosensory area; 
	SII, secondary somatosensory area; 
	STs, superior temporal sulcus; 
	VIP, ventral intraparietal area. 
	IPs and Ls have been opened to show hidden areas.
	Figure and caption reproduced from~\citet{rizzolatti.etal:1996}.}
	\label{fig:actions:mirror:brain-sx}
\end{figure}
% ---------------------------------------------------------------------------- %

% ---------------------------------------------------------------------------- %
\begin{figure}
	\centering
	  \subfigure[\label{fig:actions:F5:motor:a:1}]
	  {\includegraphics[width=0.35\textwidth]{include/actions/images/motor_a1.tps}}
	  \hspace{0.05\textwidth}
	  \subfigure[\label{fig:actions:F5:motor:a:2}]
	  {\includegraphics[width=0.35\textwidth]{include/actions/images/motor_a2.tps}}


	  \subfigure[\label{fig:actions:F5:motor:a:3}]
	  {\includegraphics[width=0.35\textwidth]{include/actions/images/motor_a3.tps}}
	  \hspace{0.05\textwidth}
	  \subfigure[\label{fig:actions:F5:motor:a:4}]
	  {\includegraphics[width=0.35\textwidth]{include/actions/images/motor_a4.tps}}


	\caption[F5 Motor Neuron]{\textbf{F5 Motor Neuron.}
	An example of a grasping neuron is shown. 
	It discharges during precision grip, regardless of the effector involved:
	(a) right hand, (b) left hand.
	When the monkey performs whole hand grasping movements the neuron is silent.
	Moreover, the neuron still express effector-response invariance: (c) right
	hand, (d) left hand.
	Reproduced from~\citet{metta.etal:2006}.}
	\label{fig:actions:F5:motor:a}
\end{figure}
% ---------------------------------------------------------------------------- %

% ---------------------------------------------------------------------------- %
\subsection{Motor neurons}
\label{sec:actions:F5:motor}
% ---------------------------------------------------------------------------- %
In 1986 Kurata and Tanji examined single unit activity in the premotor cortex of
three awake and behaving macaque monkeys. 
The authors found that most of the neurons in F5 discharge during active
movements of the hand or mouth, or both~\citep{kurata.tanji.1986}.
Two years later Rizzolatti and colleagues found that in most F5 neurons the
discharge correlates with a goal-directed action rather than with the individual
movements that form it~\citep{rizzolatti.etal:1988}.
As a consequence F5 neurons were classified into various categories
corresponding to the triggering action.
The most common categories are: grasping, manipulating, tearing and holding
neurons.
Furthermore, hand related neurons activation is not reported for hand movements
similar, but different in goal, to those effective in triggering them (e.g.,
scratching, pushing away).
The majority of them are active during movements that share a particular goal,
regardless of the effector used to attain them.
Specificity is reported for different categories of grasp: precision grip,
finger prehension and  whole hand grasping.
Moreover, F5 neurons specificity emerges also within the boundaries marked by a
single category of grip.
For example, whole hand grasping of a sphere and whole hand grasping of a
cylinder activate different units, demonstrating shape selectivity in the
response.

Motor neurons in F5 show time-dependent properties. Some of them are
active during the whole action they code, some discharge during atomic
movements such as fingers opening and closure, some just when the contact
between the hand and the object is established.

According to~\citet{metta.etal:2006,fadiga.etal:PRESS}, F5 forms a 
``vocabulary'' of motor actions where each word is represented by the activation
of a population of neurons.
The meaning of each word is tightly linked with the motor action coded by a
specific population of neurons. 
Furthermore, action are described in general terms (e.g.: hold, take and
tear); other words are used to specify how actions are accomplished (e.g.:
precision grip, finger prehension, whole hand prehension) or what effector is
involved in the motor action (e.g.: finger bending and extension). 
% ---------------------------------------------------------------------------- %
\subsection{Visuomotor neurons}
\label{sec:actions:F5:visuomotor}
% ---------------------------------------------------------------------------- %
Area F5 is somatotopically organized. In addition, F5 is subdivided in two main
histochemical sectors~\citep{rizzolatti.etal:1988}.
The first one is buried inside the arcuate sulcus whereas the second one is
located on the cortical convexity.
Neurons found in both sectors discharge when the monkey performs hand and mouth
motor actions. 
Furthermore, F5 neurons share the same motor
properties (Section~\ref{sec:actions:F5:motor}).

Many F5 cells also discharge in response to visual stimuli. 
However, neurons from different sectors require unlike visual
stimuli in order to fire. 
For instance, ``canonical'' neurons in the arcuate sulcus discharge when a
three dimensional object is presented even though a strong correlation between
the presented object a particular kind of grip must exist.
Neurons located in the cortical convexity, the novel ``mirror neurons'', 
discharge when the macaque monkey observes hand actions performed by another 
monkey or even by an experimenter.

Canonical and mirror neurons provide a link between action production and 
perception, generating an ``internal copy of a potential
action'', quoting~\citet{rizzolatti.fadiga:1998}.
According to the authors, F5 constitutes the neural basis for understating
actions made by others.
% ---------------------------------------------------------------------------- %
\subsubsection{Canonical neurons}
\label{sec:actions:F5:visuomotor:canonical}
% ---------------------------------------------------------------------------- %
The visuomotor properties of F5 neurons have been extensively examined 
by~\citet{murata.etal:1997} during ``grasping in light'', ``object fixation''
and ``grasping in dark'' motor tasks.
The experiments have been carried on using a behavioral paradigm in which
object-related visual responses are dissociated from motor responses
directed to the same object.
Moreover, the recorded neurons discharge is analyzed during  the phase between
object presentation and the onset of movement and during movement execution.

In this context the experimenters found that F5 neurons respond selectively to
different visual stimuli (that is, the presentation of three dimensional
objects) even in absence of a subsequent motor action (that is, grasping the
presented object).

A different behavioral procedure characterize each motor task. However, the
monkey was seated, the experiments took place in dark and the objects were
presented inside a box.  Furthermore, a spot of light from a red or a green LED
was projected onto the object.
The experimenters used different LED colors to allow the monkey to discriminate
what task it was asked to perform. 

During grasping in light tasks, the monkey had to fixate a red LED for a
short period of time while pressing a key.
Key press caused object illumination. 
When the LED color switched from red to green, the monkey had to release the
key, reach for and grasp the object, pull and hold it until the LED became 
red again.
During grasping in dark the light inside the box was permanently switched
off, and the procedure reflects the previous one.
During object fixation the monkey had to press the key when the green LED was 
turned on, and keep it pressed for a short period of time while fixating the
spot of light projected onto the presented object.
When the LED changed color (from green to red), the key was released.

% ---------------------------------------------------------------------------- %
\begin{figure}[htbp]
	\centering
		\epsfig{file=include/actions/images/murata_a1.tps, width=0.65\textwidth}

	\caption[Canonical neuron: grasping in light]{\textbf{Grasping in light.}
    Rasters and histograms are aligned (vertical bar) with key press (onset
	of object presentation). 
	Legend: onset of red LED (a), key press (b), onset of green LED (c), key release
	(d), onset of object pulling (e), onset of second green LED (f) and 
	object release (g).
	Reproduced from~\citet{murata.etal:1997}.}	
	\label{fig:actions:F5:murata:a:1}
\end{figure}
% ---------------------------------------------------------------------------- %

Figure~\ref{fig:actions:F5:murata:a:1} shows the activation of a selective
visuomotor neuron recorded from the macaque monkey (grasping in light).
The neuron is selective for ring shaped objects and weekly activates when a
sphere is presented. 
As a matter of fact, responses for the other objects are not relevant.
Figure~\ref{fig:actions:F5:murata:a:2} illustrates the motor response of the
visuomotor neuron recorded while the monkey was grasping in complete darkness. 
Since the visual stimulus is missing, the canonical neuron acts like
F5  ``pure'' motor neurons (Section~\ref{sec:actions:F5:motor}).
The strongest activation appears during precision grip of the ring.
Figure~\ref{fig:actions:F5:murata:a:3} shows the activation of the same neuron
during object fixation. In other words, the monkey was not allowed to grasp the
presented object, and so the response illustrated is not affected by
the motor discharge of the neuron.
Still, the neuron strongly activates when the ring is presented.

% ---------------------------------------------------------------------------- %
\begin{figure}[htbp]
	\centering
		\epsfig{file=include/actions/images/murata_a2.tps, width=0.65\textwidth}

	\caption[Canonical neuron: grasping in dark]{\textbf{Grasping in dark.}
	Legend: same as Figure ~\ref{fig:actions:F5:murata:a:1}.
	Reproduced from~\citet{murata.etal:1997}.}	
	\label{fig:actions:F5:murata:a:2}
\end{figure}
% ---------------------------------------------------------------------------- %

With respect to the presented results, it is clear that a strict congruence
between visual and motor properties of F5 canonical neurons exists.
Neurons that discharge during the fixation of small object (e.g.: ring), also
discharge during precision grip. 
According to~\citet{fadiga.etal:PRESS}, the most likely interpretation for
visual discharge in these visuomotor neurons is that there is a close link 
between the most common three dimensional stimuli and the actions necessary to
interact with them, at least in adults.

% ---------------------------------------------------------------------------- %
\begin{figure}[htbp]
	\centering
		\epsfig{file=include/actions/images/murata_a3.tps, width=0.65\textwidth}

	\caption[Canonical neuron: object fixation]{\textbf{Object fixation.}
    Legend: onset of green LED (a), key press (b), onset of red LED (c),
	key release (d).
	Reproduced from~\citet{murata.etal:1997}.}	
	\label{fig:actions:F5:murata:a:3}
\end{figure}
% ---------------------------------------------------------------------------- %

% ---------------------------------------------------------------------------- %
\subsubsection{Mirror neurons}
\label{sec:actions:F5:visuomotor:mirror}
% ---------------------------------------------------------------------------- %
In terms of motor properties, mirror neurons are indistinguishable from F5
canonical neurons.
Regarding visual properties, mirror neurons constitute a class apart.
As a matter of fact, mirror neurons are F5 visuomotor neurons that discharge
when the monkey performs a particular goal-oriented action and when it sees
another monkey (or the experimenter) performing a similar
action~\citep{rizzolatti.fadiga:1998}.
Those properties are illustrated in Figure~\ref{fig:actions:F5:mirror:a:1}
and~\ref{fig:actions:F5:mirror:a:3}.

Mirror neurons discharge only when the monkey perceives a action characterized
by the interaction of a biological effector (hand or mouth) and an object. 
The presentation of an object alone, of an agent mimicking an action, or of an
individual making intransitive
gestures\footnote{Intransitive gestures: non-object directed gestures} are all
ineffective~\citep{fadiga.etal:PRESS}.
Moreover, the presentation of actions made by tools but similar to those made by
hands, do not evokes a response, or evoke a really weak one as shown in 
Figure~\ref{fig:actions:F5:mirror:a:2}.

The majority of F5 mirror neurons are highly specific, coding not only the
action aim but also how the action is executed. 
The most common triggering actions is grasping, followed by placing,
manipulating and holding.
Furthermore, mirror neurons discharge independently from the position of the
perceived stimulus. In other words, mirror neurons show visual generalization
properties.

Usually mirror neurons show congruence between the visually perceived action and
the executed one. 
For instance, a neuron is said to show ``strict congruence'' if the effective 
motor action (e.g.: precision grip) corresponds to the action that, when seen,
triggers the neuron (e.g.: precision grip).
Sometimes congruence between the executed and the perceived action is broader.
Broadly congruent mirror neurons discharge when a specific motor action is
performed by the monkey, that is, precision grip. 
Any type of hand grasping is effective in triggering broadly congruent neurons
(e.g.: whole hand grasping).
Accordingly to~\citet{rizzolatti.arbib:1998}, the motor requirements are usually
stricter then the visual ones.\\
Figure~\ref{fig:actions:F5:mirror:b} illustrates the response of an highly 
congruent mirror neuron. 
Initially the monkey observes the experimenter manipulating a piece of food, as
to break it. 
The recorded neuron discharges when the experimenter performs anti-clockwise
rotations.
In Figure~\ref{fig:actions:F5:mirror:b:2} the monkey rotates the piece of food
held by the experimenter who opposes the monkey movement making a rotation in
the opposite direction. The neuron shows the same behavior previously
documented.
Finally, the monkey grasps a piece of food using the same fingers as during
rotations. In this latter case, the neuron is silent.
This behavior reflects what expected for a strictly congruent mirror neuron.

% ---------------------------------------------------------------------------- %
\begin{figure}
	\centering
	  \subfigure[\label{fig:actions:F5:mirror:a:1}]
	  {\includegraphics[width=0.45\textwidth]{include/actions/images/mirror_a1.tps}}
	  \hspace{0.05\textwidth}
	  \subfigure[\label{fig:actions:F5:mirror:a:2}]
	  {\includegraphics[width=0.45\textwidth]{include/actions/images/mirror_a2.tps}}

	  \subfigure[\label{fig:actions:F5:mirror:a:3}]
	  {\includegraphics[width=0.45\textwidth]{include/actions/images/mirror_a3.tps}}
	  \hspace{0.50\textwidth}

	\caption[Mirror neuron: visuomotor properties]{\textbf{Mirror neuron: visuomotor
	properties.}
	Reproduced from~\citet{rizzolatti.etal:1996}.}	
	\label{fig:actions:F5:mirror:a}
\end{figure}
% ---------------------------------------------------------------------------- %

% ---------------------------------------------------------------------------- %
\begin{figure}
	\centering
	  \subfigure[\label{fig:actions:F5:mirror:b:1}]
		{\includegraphics[width=0.45\textwidth]{include/actions/images/mirror_b1.tps}}
		\hspace{0.05\textwidth}
	  \subfigure[\label{fig:actions:F5:mirror:b:2}]
		{\includegraphics[width=0.45\textwidth]{include/actions/images/mirror_b2.tps}}

	  \subfigure[\label{fig:actions:F5:mirror:b:3}]
		{\includegraphics[width=0.55\textwidth]{include/actions/images/mirror_b3.tps}}
		\hspace{0.40\textwidth}

	\caption[Strictly congruent mirror neuron]{\textbf{Strictly congruent mirror neuron} 
	Reproduced from~\citet{rizzolatti.etal:1996}.}	
	\label{fig:actions:F5:mirror:b}
\end{figure}
% ---------------------------------------------------------------------------- %

%% ---------------------------------------------------------------------------- %
\begin{figure}
	\centering
	  \subfigure[\label{fig:actions:F5:mirror:c:1}]
		{\includegraphics[width=0.45\textwidth]{include/actions/images/mirror_c1.tps}}
	  \hspace{0.05\textwidth}
	  \subfigure[\label{fig:actions:F5:mirror:c:2}]
		{\includegraphics[width=0.45\textwidth]{include/actions/images/mirror_c2.tps}}

	  \subfigure[\label{fig:actions:F5:mirror:c:3}]
		{\includegraphics[width=0.45\textwidth]{include/actions/images/mirror_c3.tps}}
	  \hspace{0.50\textwidth}

	\caption{\textbf{Mirror neuron: actions performed by others.}
	Reproduced from~\citet{rizzolatti.etal:1996}.}	
	\label{fig:actions:F5:mirror:c}
\end{figure}
% ---------------------------------------------------------------------------- %

% ---------------------------------------------------------------------------- %
\subsubsection{Mouth mirror neurons}
\label{sec:actions:F5:visuomotor:mouth}
% ---------------------------------------------------------------------------- %
Regarding somatotopic properties, hand actions are mostly represented in the
upper sector of F5, while mouth neurons are localized in its 
lateral portion.
Recently,~\citet{ferrari.etal:2003} investigated the properties of neurons
activating during mouth actions, recording almost five hundred F5 neurons.
About a 35\% of the neurons recorded endow mirror properties:
ingestive and communicative actions are the visually stimuli most
effective in triggering F5 mouth neurons.
The majority (85\%) of the mirror units show selectivity for
ingestive actions and a third of those were classified as strictly congruent.
Ingestive neurons discharge when the monkey perceives the experimenter grasping
food with the mouth (or with the lips), breaking it or sucking juice out of a
syringe.
Focusing on communicative mirror neurons, the most effective actions are
lip smacking, lips protrusion, tongue protrusion, teeth chatter.
For those neurons, the effective motor action is different from the effective
observed action (broadly congruent neurons).
Furthermore, the whole set of gestures effective in triggering communicative
neurons can be classified as intransitive, since no object is required. 
However, visuomotor mirror neurons do need transitive actions
in order to discharge (Section~\ref{sec:actions:F5:visuomotor:mirror}).

According to the authors, the intransitive gesture of tongue protrusion could 
evoke in the monkey the idea of liking, successively perceived as a transitive
ingestive action. 
As it is discussed later on, \citet{umilta.etal:2001} recorded mirror neurons
that do not discharge if the monkey perceives the sight of an object alone but
that discharge when the monkey observes a partially hidden hand action, such as
grasping.
In particular, if the monkey is aware of the presence of the hidden object, 
a subset of F5 mirror neurons become active when the observed
action is directed toward the object.

A second difference between communicative and the other (hand and ingestive)
mirror neurons is the discrepancy between the effective visual stimulus
(communicative) and the effective motor action (ingestive).
This discrepancy, reported as ``rather
puzzling''~\citep{rizzolatti.craighero:2004}, provides evidence suggesting that
``communicative gestures, or at least some of them, derived from ingestive
actions in evolution'' (\emph{ibid}).


% ---------------------------------------------------------------------------- %
\subsection{Action understanding}
\label{sec:actions:understanding}
% ---------------------------------------------------------------------------- %
%The mirror neuron system is a mechanism of great evolutionary importance. 
As discussed earlier in this Section, a class of F5 neurons code goal-related
motor actions such as hand grasping and mouth ingestive or communicative
actions.
The so called mirror neurons endow both motor and visuomotor properties,
discharging, for example, when the monkey acts on an object or when the monkey
sees a conspecific performing a similar task.
Moreover,mirror neurons' activity does not emerge
when the monkey observes an object alone, even if interesting from the 
monkey's point of view (e.g.: food).
%The mirror neuron system allows primates to understand actions made by other
%individuals~\citep{rizzolatti.etal:2001}.
The primitive action recognition circuitry in primates evolved in a far more 
complex system that constitutes the basis of imitation in humans, gaining great
evolutionary importance~\citep{rizzolatti.etal:2001,jannerod:2004}.
%According to~\citet{jannerod:2004} the mirror neuron system constitutes the
%basis for imitation in primates. 
As said earlier, imitation is not a primitive cognitive
function. Furthermore, only humans (and, maybe,
apes\footnote{Apes: members of the \emph{Hominoidea} superfamily of
primates. There are two families of \emph{hominoids}: the family 
\emph{Hylobatidae} consists of 4 genera and 12 species of gibbons 
(lesser apes). The family \emph{Hominidae} consists of gorillas, chimpanzees, 
orangutans and humans (great apes).}) can learn a new action
from seeing it done.
For this reason the primary cognitive function of mirror neurons is action
understanding, and not
imitation~\citep{rizzolatti.etal:2001,rizzolatti.craighero:2004}.

Mirror neurons provide a neurophysiological link between actions performed 
by an individual with actions done by others.
Neurons that discharge during actions done by others have also been found in the
cortex of the superior temporal sulcus. 
Those neurons discharge during the observation of hand movements or even when
the monkey observes another individual walking, turning the head, bending the 
torso or moving the arms~\citep{rizzolatti.craighero:2004}.
Among visual properties, STS neurons do not discharge when the monkey performs
an action, and for this reason those neurons could not be classified as mirror
neurons since they lack of pure motor properties.
According to~\citet{rizzolatti.etal:2001}:
\begin{quote}
The ``direct matching hypothesis'' [\ldots] holds that we understand
actions when we map the visual representation of the observed action onto our
motor representation of the same action.
According to this view, an action is understood when its observation causes
the motor system of the observer to ``resonate''. 
\end{quote}
In other words, an action made by another individual is understood when its
observation activate the corresponding observer's motor representation.
What is said by Rizzolatti relies on the physiological properties of mirror
neurons. 
%discharging when the monkey performs an action and when the monkey sees an 
%action done by a conspecific.
The induced visuomotor representation  
matches the motor representation evoked during active action, thus providing a 
link between action execution and action perception.
%: mirror neurons transform visual information in knowledge.
In this context, perception and execution of motor actions rely
%(?at least, in part?) 
on a common neurophysiological circuit, that is the mirror neuron
system.
% ---------------------------------------------------------------------------- %
%\subsubsection{Mirror neurons and ``hidden'' actions}
\subsubsection{Mirror neurons and partially occluded actions}
\label{sec:actions:umilta}
% ---------------------------------------------------------------------------- %
\citet{umilta.etal:2001} used a behavioral paradigm which allowed recording the
response of a subset of F5 neurons during full-vision and also during 
partially-hidden action presentation.
In the first case, the monkey observed the experimenter grasping an object after
moving his hand toward the object itself (``full vision'' condition).
In the latter case, the so called ``hidden'' condition, the final part of the
action is hidden and can only be guessed by the monkey
(e.g.: grasping the object).
The experimenters relied on an initial hypothesis (subsequently proved)
stating that visual representation is not necessary to recognize the goal of 
an action:
%\footnote{Quoting \citet{umilta.etal:2001}: 
\begin{quote}
If mirror neurons represent the neural substrate for action
recognition, they (or a subset of them) should become active also during the
observation of partially hidden actions.
\end{quote}
%}. 
More than two hundred F5 neurons were recorded; about an half of
them expressed the visuomotor properties typical of mirror neurons and 
have been tested in ``full vision'' and ``hidden'' conditions.
An half of the tested neurons discharged in both experimental conditions.
Furthermore, half of them showed a stronger response in the hidden condition,
while the other half did not show any particular difference between the testing
conditions.

% ---------------------------------------------------------------------------- %
\begin{figure}
	\centering
	  \subfigure[\label{fig:actions:understanding:umilta:a:1}]
	  {\includegraphics[width=0.3\textwidth]{include/actions/images/umilta_a1.tps}}
	  \hspace{0.025\textwidth}
	  % ---------------------------------------------------------------------- %
	  \subfigure[\label{fig:actions:understanding:umilta:a:2}]
	  {\includegraphics[width=0.3\textwidth]{include/actions/images/umilta_a2.tps}}


	  \subfigure[\label{fig:actions:understanding:umilta:a:3}]
	  {\includegraphics[width=0.3\textwidth]{include/actions/images/umilta_a3.tps}}
	  \hspace{0.025\textwidth}
	  % ---------------------------------------------------------------------- %
	  \subfigure[\label{fig:actions:understanding:umilta:a:4}]
	  {\includegraphics[width=0.3\textwidth]{include/actions/images/umilta_a4.tps}}


	\caption[Mirror neuron: action recognition]{\textbf{Mirror neuron: action recognition.}
	In figure (a) and (b) the experimenter hand starts from a fixed position and moves 
	toward an object.  
	In figure (c) and (d) the object is absent and the experimenter mimics the
	grasping action.
	Figure (a) and (c): full vision condition. Figure (b) and (d): hidden
	condition.
	Reproduced from~\citet{umilta.etal:2001}.}
	\label{fig:actions:understanding:umilta:a}
\end{figure}
% ---------------------------------------------------------------------------- %

Figure~\ref{fig:actions:understanding:umilta:a} is an example of a mirror
neuron that discharge during action observation in full vision and in hidden 
condition but not when the object is absent (mimed or intransitive action). 
During the hidden condition, the recorded neuron discharged as if there were no
screen.
Additionally, the neuron did not discharge when the object was absent,
although the presented action was identical to the previous one in hidden
condition.
% ---------------------------------------------------------------------------- %
\subsubsection{Audio-visual mirror neurons}
\label{sec:actions:kohler}
% ---------------------------------------------------------------------------- %
\citet{kohler.etal:2002} recorded neurons in monkey premotor cortex that 
discharge when the monkey performs an action and when it hears the related
sound.
The most represented action are object breaking and paper ripping. 
Twenty-nine out of thirty-three recorded neurons endow auditory selectivity
during randomly presented hand actions.
The experimental protocol consisted in vision and sound, sound-only, 
vision-only and motor-only conditions.

Figure~\ref{fig:actions:understanding:kohler:a} shows two examples of
F5 audio-visual mirror neurons.
Those mirror neurons discharges when the monkey observes
an action hearing the related sound (V+S). 
Two cases are presented: paper ripping (left) and stick dropping (right).
The sound of the action performed out of the monkey's sight is equally
effective in activating the two neurons (S) while control-stimuli such as white 
noise (CS$_1$) and monkey call (CS$_2$) do not lead to activation.
%(?che schifo?).
%The second neuron is selective for ``dropping stick'' actions but its behavior
%is similar to what previously discussed.
% ---------------------------------------------------------------------------- %
\begin{figure}[htbp]
	\centering
		\epsfig{file=include/actions/images/kohler_a.tps, width=0.45\textwidth}

	\caption[Audio-visual mirror neurons]{\textbf{Audio-visual mirror neurons.}
	Two examples of monkey F5 audio-visual neurons. 
	Label at the upper left of each panel: V+S, visual and auditive stimulus; 
	S, auditive stimulus; CS$_{1}$ and CS$_{2}$, auditive stimulus
	(control condition).
	Reproduced from~\citet{kohler.etal:2002}.}
	\label{fig:actions:understanding:kohler:a}
\end{figure}
% ---------------------------------------------------------------------------- %

% ---------------------------------------------------------------------------- %
\section{Human mirror neurons}
\label{sec:actions:human}
% ---------------------------------------------------------------------------- %
Mirror neurons are considered one of the most important discoveries of the
last decade:
\begin{quote}
I predict that mirror neurons will do for psychology what DNA did for biology:
they will provide a unifying framework and help explain a host of mental 
abilities that have hitherto remained mysterious and inaccessible to 
experiments.
\end{quote}
What is said here by V. S. Ramachandran\footnote{V. S., Ramachandran. 
Mirror neurons and imitation learning as the driving force behind the 
great leap forward inhuman evolution.
%http://www.edge.org/3rd\_culture/ramachandran/ramachandran\_p1.html.
http://www.edge.org/3rd\_culture/ramachandran\_p1.html.
Retrieved Apr. 26, 2007.} 
accounts for the ongoing research on the mirror system in humans.
In fact, mirror neurons are important for understanding the actions made by
others and for imitation learning~\citep{rizzolatti.craighero:2004}.
Possible implication of the mirror system in the ``Theory of Mind''
have been speculated\footnote{M. Arbib. The Mirror System Hypothesis.
Linking Language to Theory of Mind, 2005. http://www.interdisciplines.org/coevolution/papers/11. Retrieved Apr. 12, 2007.}
and several studies reveal that the mirror system could be involved in speech 
perception~\citep{rizzolatti.arbib:1998,fadiga.etal:PRESS}. 

Direct evidence for a mirror system in humans is lacking since no studies
recorded single unit activity.
On the other hand, numerous studies provide results that, indirectly, confirm 
that such system exists in humans\footnote{A first evidence goes back to the
50s. In particular
\citet{cohen-seat.etal:1954} reported that a desynchonization of an EEG rhythm 
recorded from the central derivations occurs when humans perform active
movements and when they see actions done by others.
More recent experiments confirm what reported by Cohen recording primary motor
cortex activity via EEG~\citep{cochin.etal:1998,cochin.etal:1999} and 
MEG~\citep{hari.etal:1998}.} and that it develops in infants between six and
twelve months of age~\citep{falck-ytter.etal:2006}.

\citet{grafton.etal:1997} used PET to investigate premotor areas activation
in right-handed normal subjects during the observation of graspable objects.
The experimenters asked the subjects to silently name the tools
and silently name their use.
Object observation activates the left dorsal premotor cortex in a sector where
hand and arm movements are represented and another portion of the cortex
localized on the border between BA45 and BA46\footnote{BA46 is 
considered, functionally speaking, as a part of the dorsolateral prefrontal 
cortex. BA44 and BA45 are located in the opercular and triangular sections of
the inferior frontal gyrus of the frontal lobe of the cortex. Furthermore, BA45
and BA46 constitute the ``Broca's Area'', a section of the human brain involved
in language processing, speech production and
comprehension~\citep{kandel.schwartz.jessel:2000}. Broca's area is
considered the homologous of F5 in the monkey.}. 
During silent naming of the object, a reinforcement of the activation of the 
premotor cortex was observed.
This result recalls what discussed earlier for canonical F5 neurons in the
monkey (Section~\ref{sec:actions:F5:motor}).
In fact, the presentation of a graspable object evoke spontaneous activity in 
the premotor cortex.

\citet{grezes.decety:2002} used PET to verify if the perception of objects 
irrespective of the task required to the subject and the perception of
non-objects are correlated in terms of activated cortical
sectors\footnote{See also: \citet{grezes.etal:1998}}.
The authors found a set of cortical regions in the left hemisphere that become
active during all the tested experimental conditions, which are
the occipito-temporal junction, the inferior parietal lobule, 
the SMA-proper\footnote{SMA: Supplementary Motor Area. SMA is a part of the
sensorimotor cerebral cortex, and it is located on each side of the central
sulcus. SMA is divided in two distinct parts: SMA-proper and pre-SMA.
Human SMA-proper is the analogous of area F3 in monkeys while pre-SMA is the 
analogous of area F6~\citep{kandel.schwartz.jessel:2000}.},
the dorsal and ventral precentral gyrus and Broca's area
(Figure~\ref{fig:appendix:lbrain-side} and~\ref{fig:appendix:lbrain-top}).
Besides this common set of cortical regions,
the activation of a specific region was associated with each task.
According to~\citet{fadiga.etal:PRESS}, those results confirm that 
``the mere perception of objects activates representation of possible object 
affordances and the motor plans associated with their execution''.

During the last decade, numerous studies provided indirect evidence for a mirror
system in humans. 
The first evidence for such a system was provided recording 
MEPs\footnote{MEP: motor evoked potential.} after stimulating the motor cortex
of normal human subjects with non-invasive
TMS\footnote{TMS: transcranical magnetical
stimulation.}~\citep{fadiga.etal:1995}.
For instance, the stimulation of the human motor cortex at the level where hand
is mapped, activates the contralateral hand muscles\footnote{Private
demonstration by L. Craighero and P. Senot (April 2007).}.
The behavioral context modulates MEPs' amplitude, thus providing
an effective method to study the human cortex.

Regarding the experiments carried on by~\citet{fadiga.etal:1995}, the left 
motor cortex of normal human subjects was stimulated 
while the subjects observed transitive hand actions (e.g.: the experimenter 
grasped a 3D-object) and intransitive arm movements (e.g.: the experimenter 
traced geometrical figures in the air).
Furthermore, observation of three dimensional objects and the detection of the
dimming a spot of light were used as control conditions.
The experimenters found that MEPs' facilitation occurs during observation of
transitive and intransitive gestures with respect to the control conditions.
The results demonstrate that the excitability of the motor system increases when
a subject observes a conspecific performing hand and arm actions.
Moreover, the observed MEP facilitation involved only those muscles used by the
observed subjects to perform the selected actions.
Additionally, the observation of the sights of an object alone was not
sufficient to modulate the excitability of the motor cortex.

The presented results provide direct evidence for the hypothesis that 
part of the human motor cortex acts as a resonator thus matching action
production and action observation.
Moreover, the observation of ``meaningless'' intransitive actions
activates the human mirror system while the monkey's mirror neurons discharge
only during the observation of action directed toward objects.
On the other hand, the experiment does not reveal details about the mirror
system circuitry in humans.
\citet{grafton.etal:1996} found that observation of grasping actions activates 
the left hemisphere at the level of the superior temporal sulcus, 
the inferior parietal lobule and the inferior frontal gyrus (BA45), indicating
that both humans and monkeys have a system involved in action recognition.

Still another difference exists between the monkey and the human mirror
system. 
As said before, monkeys are not good imitators, while the contrary is true for
humans (Section~\ref{sec:actions:understanding}). 
The existing links between the mirror neuron system and the imitational
framework that accomunates normal humans is discussed in the next Section.
%(Section~\ref{sec:actions:imitation}).
A characteristic of the human mirror system seems to be its ``granularity'',
intended as the ability to recognize not only particular classes of actions
(e.g.: precision grip, whole hand prehension) but also their constituent
movements.
In fact, while the monkey mirror neurons discharge during the observation of
an action, the human mirror neurons discharge also during the observation of
the movements forming the action itself~\citep{rizzolatti.craighero:2004}.
According to this point of view, mirror neurons that discharge both during
transitive and intransitive actions could provide the neural basis for action
imitation in humans.
Moreover, the ability to extract action constituents from complex actions is a
crucial for action recognition in humans.
%On the other hand, mirror neurons located in monkey area F5 discharge during 
%the observation of transitive actions toward objects. 
The question that arises regards the degree of specialization gained by Broca's
area precursor in extracting meaningless actions from complex gestures.

A study by~\citep{fadiga.etal:2006} found that the human mirror system 
activates not only during the observation of transitive and intransitive
hand/harm actions, but also when the subject looks at ``hand shadows'' mimicking
animals opening their mouth. 
%The experimental paradigm was prepared to inspect the existing
%relationship between the activation of the mirror system and the level of 
%details characterizing a certain hand action.
Hand shadows, formed by intransitive finger movements, constitute a subset of
hand actions that only implicitly contain the representation of human hands.
The subjects where scanned using fMRI while watching at different stimuli
(videos) presented on a screen. 
The stimuli represented hand shadows mimicking animals opening
their mouth, real animals opening their mouth and intransitive random finger 
movements.
Furthermore, the static version (a video frame) of each stimulus was presented.
The activations recorded during the presentation of the videos were normalized
using the activations due the static stimuli in order to ``emphasize the action 
component of the gesture'' (\emph{ibid}).
The experimenters found that only hand shadows, and not real animals, activate 
Broca's area (BA44 and BA46).
The experiment by Fadiga and coworkers demonstrate that the human mirror system
not only resonates during meaningless hand and finger gestures, but it is also 
involved in the extraction of ``action units'' starting from complex
gestures.
%that overlap in time and in space.
Historically, Broca's region is said to be involved in both perception and 
production of speech~\citep{kandel.schwartz.jessel:2000}.
Considering the fact that Broca's region is the homologue of monkey F5, the
authors hypothesize that its precursor was involved in ``generating/extracting
action meanings by organizing/interpreting motor sequences in terms of goal'',
thus addressing the vocabulary of both speech and hand actions (\emph{ibid}).
% ---------------------------------------------------------------------------- %
\subsubsection{Mirror neurons and imitation}
\label{sec:actions:imitation}
% ---------------------------------------------------------------------------- %
The previous Sections covered the topic of mirror neurons in monkeys and in 
humans.
Recently, quite a few papers have been published regarding the role of mirror
neurons and empathy~\citep{wicker.etal:2003} or
autism~\citep{oberman.etal:2005}.
However, the aim of this dissertation drifts away from presenting the mirror 
neuron system at such an high level of detail.
In this respect, before introducing the relationship between the mirror system
and speech perception, a last paragraph is spent talking about how mirror
neurons onstitute the neural foundations for imitation in humans.

Action imitation requires action observation, meaningful information extraction
and then motor execution of the observed action. 
According to the direct matching hypothesis, the mirror system links action 
perception and action production, translating the observed action in the 
corresponding internal representation~\citep{rizzolatti.etal:2001}.
An fMRI experiment was used to study normal subjects observing and imitating a 
specific hand action, and to test the plausibility of the direct matching
hypothesis~\citep{iacoboni.etal:1999}.
As a control condition, the subjects were asked to simply observe, on a
computer screen, a finger moving, a cross on a static finger or a cross on a
solid color background.
The previous conditions were used also for the imitation task, but the subjects
had to lift one of his right fingers in response to the visual stimuli.
During all trials, the subjects were aware about the fact that they had to move
or not to move a particular finger. In this context, motor imagery of the finger
is present during both observation and imitation (observation-execution) tasks.
%This bg activity was potentiated when the stimulus to be imitated was present.
The authors found, as expected, that the left frontal operculum (BA44) and
the right anterior parietal cortex were more active during imitation tasks than
during simple observation.
Moreover, between imitation tasks, the activation of the regions above was
much higher when the subject observed the moving fingers than in any other case
involving the cross, thus proving that the left frontal operculum and the right
anterior parietal cortex endow an imitation mechanism that agrees with the
direct matching hypothesis.
% ---------------------------------------------------------------------------- %
