% ---------------------------------------------------------------------------- %
\begin{figure}
	\centering
	\subfigure[\label{fig:linguometer:architecture:sig:lmwords2:1}]
	{\includegraphics[width=0.25\textwidth]{include/linguometer/images/lmwords_1.tps}}
	\hspace{0.05\textwidth}
	\subfigure[\label{fig:linguometer:architecture:sig:lmwords2:2}]
	{\includegraphics[width=0.25\textwidth]{include/linguometer/images/lmwords_2.tps}}
	\hspace{0.05\textwidth}
	\subfigure[\label{fig:linguometer:architecture:sig:lmwords2:3}]
	{\includegraphics[width=0.25\textwidth]{include/linguometer/images/lmwords_3.tps}}

	\subfigure[\label{fig:linguometer:architecture:sig:lmwords2:4}]
	{\includegraphics[width=0.25\textwidth]{include/linguometer/images/lmwords_4.tps}}
	\hspace{0.05\textwidth}
	\subfigure[\label{fig:linguometer:architecture:sig:lmwords2:5}]
	{\includegraphics[width=0.25\textwidth]{include/linguometer/images/lmwords_5.tps}}
	\hspace{0.05\textwidth}
	\subfigure[\label{fig:linguometer:architecture:sig:lmwords2:6}]
	{\includegraphics[width=0.25\textwidth]{include/linguometer/images/lmwords_6.tps}}
	
	\caption[Stimuli-presentation program states]{\textbf{Stimuli-presentation 
	program states}: \emph{lmwords} passes through four distinct states: (a)
	\emph{pause}, (b) \emph{ready}, (c,d) \emph{word presentation} and
	(e) \emph{done}. The last panel (f) shows how the stimuli are presented on 
	the LCD screen used during the recording sessions.
	During each state-transition, the screen remains black for half a second.
	Furthermore, the subjects are aware of meaning of each state. 
	In fact, the subjects can freely talk to the experimenters until the program
	enters in the ``pause'' state and after the program enters the ``end''
	state.}
	\label{fig:linguometer:architecture:sig:lmwords}
\end{figure}
% ---------------------------------------------------------------------------- %
