% ---------------------------------------------------------------------------- %
\begin{figure}
	\centering
	  \subfigure[\label{fig:linguometer:technical:glasses:0}]
		{\includegraphics[width=0.25\textwidth]{include/linguometer/images/glasses_0.tps}}
		\hspace{0.05\textwidth}
	  \subfigure[\label{fig:linguometer:technical:glasses:1}]
		{\includegraphics[width=0.25\textwidth]{include/linguometer/images/glasses_1.tps}}
		\hspace{0.05\textwidth}
	  \subfigure[\label{fig:linguometer:technical:glasses:2}]
		{\includegraphics[width=0.25\textwidth]{include/linguometer/images/glasses_2.tps}}
	\caption[Head movement compensation glasses]{\textbf{Head movement
	compensation glasses}: (a) position of the reference sensors (10, 11 and 12)
	and of the upper teeth sensor. Being the teeth constraint to the head,
	measuring the ISD values between sensor 6 and the sensors on the glasses
	turns out to be useful for detecting the motion of the latter. It may happen
	that the glasses wear by the subject slide towards the tip of the nose (b
	and c). Refer to Figure~\ref{fig:experiments:map} for a complete sensor
	map.}
	\label{fig:linguometer:technical:glasses}
\end{figure}
% ---------------------------------------------------------------------------- %
