% ---------------------------------------------------------------------------- %
%\comment{
\chapter*{ }
  \vspace{180pt}
  \begin{minipage}[h!]{\textwidth}
  \begin{center}
  This page in intentionally left blank
  \end{center}
\end{minipage}

\chapter*{ }
  \vspace{180pt}
  \begin{minipage}[h!]{\textwidth}
  \begin{center}
  This page in intentionally left blank
  \end{center}
\end{minipage}
%}
% ---------------------------------------------------------------------------- %
\comment{
\chapter*{Sommario}
% ---------------------------------------------------------------------------- %
Il progetto CONTACT (NEST Contract No 5010) mira ad investigare i meccanismi
senso-motori comuni sia alla percezione sia alla produzione del parlato e della
manipolazione.

Liberman sostiene che i veri oggetti del parlato non sono i suoni,
bens\'i i gesti articolatori alla base della loro produzione, rappresentati a
livello corticale come comandi motori invarianti~\citep{liberman.mattingly:1985}.
Durante l'ultimo decennio, gli studi nel settore della neurofisiologia si sono
concentrati sull'idea che, a livello corticale, le azioni non sono rappresentate
in termini procedurali ma come obbiettivi (``goal'').
Nella scimmia, un insieme di cellule localizzate a livello dell'area F5 (neuroni
specchio) si attiva sia durante l'esecuzione sia durante l'osservazione di
azioni transitive, collegando cos\`i i meccanismi di produzione ai percettivi~\citep{rizzolatti.etal:1988,rizzolatti.etal:1996,rizzolatti.fadiga:1998}.
Nei bambini, linguaggio e manipolazione si sviluppano parallelamente~\citep{lennenberg:1967,kandel.schwartz.jessel:2000}, inoltre i risultati presentati nel
Capitolo~\ref{ch:speech} suggeriscono l'esistenza di una connessione tra le aree
dedicate al linguaggio e le aree dedicate al controllo motorio, specialmente a
livello dell'area di Broca~\citep{fadiga.etal:PRESS}.
Le caratteristiche dei neuroni specchio sono descritte nel 
Capitolo~\ref{ch:actions}.

Nel Capitolo~\ref{ch:speech} si introducono i temi generali del linguaggio, 
focalizzandosi specialmente sul modello di Wernicke-Geschwind e sullo studio 
delle afasie.
Inoltre, si introducono le strutture anatomiche deputate alla sua produzione (es:
polmoni, lingua, labbra e laringe).
Verso la fine del Capitolo~\ref{ch:speech} sono presentati i recenti studi che 
collegano le teorie di~\citet{liberman.mattingly:1985} con il sistema dei
neuroni specchio~\citep{rizzolatti.etal:1988}.

I tre capitoli a seguire presentano il lavoro svolto per il progetto CONTACT
durante l'integrazione e l'utilizzo di un setup sperimentale devoluto
all'acquisizione simultanea di parametri fono-articolatori, chiamato
informalmente ``Linguometro''\footnote{``Linguometro'', ovvero ``misura-lingua''
\`e stata coniata dal Prof. Giulio Sandini}.
Il progetto CONTACT mira a realizzare una mappa audio-motoria (AMM, audio-motor
map) con lo scopo di proiettare lo spazio delle feature acustiche sullo spazio
delle feature motorie, in modo analogo a quanto \`e stato realizzato con
successo da~\citet{metta.etal:2006} realizzando una mappa visuo-motoria (VMM,
visuo-motor map) col fine di migliorare le prestazioni di un sistema per il
riconoscimento automatico di gesti manipolatori.
Una volta realizzata la AMM, sar\`a possibile realizzare un riconoscitore di
linguaggio Bayesiano che ne sfrutti le potenzialit�.
Al fine di realizzare la AMM, � necessario acquisire un gran numero di feature
fono-articolatorie misurando la configurazione motoria degli apparati deputati
all'articolazione durante la produzione del parlato.

Misurare l'attivit\`a degli apparati deputati all'articolazione del parlato \`e
un compito complicato, soprattutto a causa della bassa accessibilit\`a di queste
strutture.
Secondariamente, non esistono strumenti dedicati sul mercato.
Partendo da questi due presupposti, l'autore di questa Tesi ha connesso
molteplici strumenti, ha realizzato strutture di supporto alle registrazioni ed
infine ha scritto il software necessario sia al controllo del setup
sperimentale, sia all'elaborazione del dataset.
In Sezione~\ref{ch:linguometer:instrumentation} l'autore descrive la
strumentazione utilizzata durante le registrazioni con il Linguometro.
In Sezione~\ref{ch:linguometer:architecture} si descrive nel dettaglio 
l'architettura del setup ed il processo che ha portato alla sua integrazione,
mentre in Sezione~\ref{ch:linguometer:technical} si discutono alcuni aspetti
prettamente tecnici.

Ultimata l'integrazione del Linguometro, sono stati registrati nove parlanti
diversi (Capitolo~\ref{ch:experiments}).
Nel Capitolo~\ref{ch:results} l'autore fornisce
un esempio dei dati acquisiti e una descrizione del toolkit \emph{LMTools2}
scritto ed utilizzato per processare i dati acquisiti (allineamento e
segmentazione).
Al termine di questa Tesi, sono tratte le conclusioni circa il lavoro svolto per
il progetto CONTACT (Capitolo~\ref{ch:conclusions}).
}
% ---------------------------------------------------------------------------- %

