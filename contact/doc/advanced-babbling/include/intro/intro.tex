% ---------------------------------------------------------------------------- %
%%% ---------------------------------------------------------------------------- %
%\comment{
\chapter*{ }
  \vspace{180pt}
  \begin{minipage}[h!]{\textwidth}
  \begin{center}
  This page in intentionally left blank
  \end{center}
\end{minipage}

\chapter*{ }
  \vspace{180pt}
  \begin{minipage}[h!]{\textwidth}
  \begin{center}
  This page in intentionally left blank
  \end{center}
\end{minipage}
%}
% ---------------------------------------------------------------------------- %
\comment{
\chapter*{Sommario}
% ---------------------------------------------------------------------------- %
Il progetto CONTACT (NEST Contract No 5010) mira ad investigare i meccanismi
senso-motori comuni sia alla percezione sia alla produzione del parlato e della
manipolazione.

Liberman sostiene che i veri oggetti del parlato non sono i suoni,
bens\'i i gesti articolatori alla base della loro produzione, rappresentati a
livello corticale come comandi motori invarianti~\citep{liberman.mattingly:1985}.
Durante l'ultimo decennio, gli studi nel settore della neurofisiologia si sono
concentrati sull'idea che, a livello corticale, le azioni non sono rappresentate
in termini procedurali ma come obbiettivi (``goal'').
Nella scimmia, un insieme di cellule localizzate a livello dell'area F5 (neuroni
specchio) si attiva sia durante l'esecuzione sia durante l'osservazione di
azioni transitive, collegando cos\`i i meccanismi di produzione ai percettivi~\citep{rizzolatti.etal:1988,rizzolatti.etal:1996,rizzolatti.fadiga:1998}.
Nei bambini, linguaggio e manipolazione si sviluppano parallelamente~\citep{lennenberg:1967,kandel.schwartz.jessel:2000}, inoltre i risultati presentati nel
Capitolo~\ref{ch:speech} suggeriscono l'esistenza di una connessione tra le aree
dedicate al linguaggio e le aree dedicate al controllo motorio, specialmente a
livello dell'area di Broca~\citep{fadiga.etal:PRESS}.
Le caratteristiche dei neuroni specchio sono descritte nel 
Capitolo~\ref{ch:actions}.

Nel Capitolo~\ref{ch:speech} si introducono i temi generali del linguaggio, 
focalizzandosi specialmente sul modello di Wernicke-Geschwind e sullo studio 
delle afasie.
Inoltre, si introducono le strutture anatomiche deputate alla sua produzione (es:
polmoni, lingua, labbra e laringe).
Verso la fine del Capitolo~\ref{ch:speech} sono presentati i recenti studi che 
collegano le teorie di~\citet{liberman.mattingly:1985} con il sistema dei
neuroni specchio~\citep{rizzolatti.etal:1988}.

I tre capitoli a seguire presentano il lavoro svolto per il progetto CONTACT
durante l'integrazione e l'utilizzo di un setup sperimentale devoluto
all'acquisizione simultanea di parametri fono-articolatori, chiamato
informalmente ``Linguometro''\footnote{``Linguometro'', ovvero ``misura-lingua''
\`e stata coniata dal Prof. Giulio Sandini}.
Il progetto CONTACT mira a realizzare una mappa audio-motoria (AMM, audio-motor
map) con lo scopo di proiettare lo spazio delle feature acustiche sullo spazio
delle feature motorie, in modo analogo a quanto \`e stato realizzato con
successo da~\citet{metta.etal:2006} realizzando una mappa visuo-motoria (VMM,
visuo-motor map) col fine di migliorare le prestazioni di un sistema per il
riconoscimento automatico di gesti manipolatori.
Una volta realizzata la AMM, sar\`a possibile realizzare un riconoscitore di
linguaggio Bayesiano che ne sfrutti le potenzialit�.
Al fine di realizzare la AMM, � necessario acquisire un gran numero di feature
fono-articolatorie misurando la configurazione motoria degli apparati deputati
all'articolazione durante la produzione del parlato.

Misurare l'attivit\`a degli apparati deputati all'articolazione del parlato \`e
un compito complicato, soprattutto a causa della bassa accessibilit\`a di queste
strutture.
Secondariamente, non esistono strumenti dedicati sul mercato.
Partendo da questi due presupposti, l'autore di questa Tesi ha connesso
molteplici strumenti, ha realizzato strutture di supporto alle registrazioni ed
infine ha scritto il software necessario sia al controllo del setup
sperimentale, sia all'elaborazione del dataset.
In Sezione~\ref{ch:linguometer:instrumentation} l'autore descrive la
strumentazione utilizzata durante le registrazioni con il Linguometro.
In Sezione~\ref{ch:linguometer:architecture} si descrive nel dettaglio 
l'architettura del setup ed il processo che ha portato alla sua integrazione,
mentre in Sezione~\ref{ch:linguometer:technical} si discutono alcuni aspetti
prettamente tecnici.

Ultimata l'integrazione del Linguometro, sono stati registrati nove parlanti
diversi (Capitolo~\ref{ch:experiments}).
Nel Capitolo~\ref{ch:results} l'autore fornisce
un esempio dei dati acquisiti e una descrizione del toolkit \emph{LMTools2}
scritto ed utilizzato per processare i dati acquisiti (allineamento e
segmentazione).
Al termine di questa Tesi, sono tratte le conclusioni circa il lavoro svolto per
il progetto CONTACT (Capitolo~\ref{ch:conclusions}).
}
% ---------------------------------------------------------------------------- %


% ---------------------------------------------------------------------------- %
\section{Introduction}
\label{ch:intro}
% ---------------------------------------------------------------------------- %
The CONTACT Project is investigating the
sensorimotor mechanisms common to both perception and production of speech and
manipulation.
%
According to Liberman's theoretical framework,
the real constituents of speech are not sounds, but the articulatory
gestures underlining it, represented in the brain as invariant motor 
commands~\citep{liberman.mattingly:1985}.
The last decade of neurophysiological studies focused on the novel idea that
actions are represented in terms of goals rather then in terms of
procedures. 
Single unit recordings of monkeys F5 neurons demonstrate that a set of cells
(\emph{mirror neurons}) discharge both during the execution of goal-directed 
transitive  actions and during the observation of a similar action, thus 
linking production to perception~\citep{rizzolatti.etal:1988,rizzolatti.etal:1996,rizzolatti.fadiga:1998}. 
Moreover, infants develop motor and language abilities in
parallel~\citep{lennenberg:1967,kandel.schwartz.jessel:2000} and the results of
studies by Fadiga et al support
%
% presented in Section~\ref{ch:speech} support
%
the hypothesis that there is a functional connection between language related
and motor-control related areas in the dominant left
hemisphere, where Broca's areas is located~\citep{fadiga.etal:PRESS}. 
%The fundamental properties of the mirror neurons are described in 
%Section~\ref{ch:actions}.

%In Section~\ref{ch:speech} the author provides a brief introduction to the topic
%of language, describing the classical Wernicke-Geschwind model of speech
%processing (Section~\ref{sec:speech:language:wg}).
%Furthermore, the articulators (e.g.: lungs, tongue, lips and larynx) are 
%described from a physiological perspective, mainly focusing on their 
%involvement in the production of the gestures of speech
%(Section~\ref{sec:speech:language:mechanism}).
%Finally, the results that account for a motor role of Broca's area in speech are
%presented in Section~\ref{sec:speech:mirror}, thus linking the theoretical
%framework by~\citet{liberman.mattingly:1985} with the mirror 
%system~\citep{rizzolatti.etal:1988}.

This document
describes the work done for the 
CONTACT Project assembling and running an experimental setup for the 
simultaneous acquisition of phono-articulatory parameters, the so-called 
Linguometer\footnote{The 
word Linguometer is an English translation of the Italian word
``Linguometro'', which in turn was invented by Prof. Giulio Sandini
(\emph{linguo}: from the Italian word ``tongue''; \emph{metro}: from Latin 
``metro'' or Greek ``metron'', meaning ``measure'').}.
This document is in large part based on a recent thesis on 
the topic~\cite{tavella2007simultaneous}.
%
%% n this context, the CONTACT Project aims to build an audio-motor map (AMM), 
%% that will translate the acoustic features to motor features, similarly to what
%% has been done for a model of the mirror system, where~\citet{metta.etal:2006}
%% built an visuo-motor map (VMM) to improve the performance of a system devoted to
%% classify hand gestures.
%% As a consequence, a Bayesian speech recognition system will be designed to 
%% take full advantage of the AMM.
%% In order to train the model, it has been necessary 
%% to acquire a large dataset of phono-articulatory parameters
%% by measuring the motor configuration of the articulators during speech 
%% production.
%
Measuring the activity of the articulators (the tongue, the larynx, the lips and
the jaw) during speech is not trivial. 
Firstly, the articulators are fairly inaccessible.
Secondly, no Linguometer-like instrumentation is available on the market.
So we have assembled the Linguometer as a ``constellation'' of
different instruments, building custom devices, and 
writing the software required to control data acquisition and to align the
dataset.
In Section~\ref{ch:linguometer:instrumentation} the devices
used for the simultaneous recordings are described, while in 
Section~\ref{ch:linguometer:architecture} the architecture of
the Linguometer is described. Section~\ref{ch:linguometer:technical}
focuses on various technical aspects faced during the integration
process.
%
Once construction of the Linguometer was complete, a total of nine
subjects were recorded while uttering words and pseudo-words.
(Section~\ref{ch:experiments}).
An example of the acquired data and a brief description of the \emph{LMTools2}
toolkit written to align and segment
the whole dataset is provided in Section~\ref{ch:results}.
Preliminary phonetic analysis of the output by IST is 
described in Section~\ref{ch:phonetic}.
%
Section~\ref{sect:babbling} describes continuing work on phonetic
imitation on a robot, with Section~\ref{sect:vowel} giving 
experimental results on deriving a good space for controlling
the production of vowel sounds.


% ---------------------------------------------------------------------------- %
\pagebreak
% ---------------------------------------------------------------------------- %
